\chapter[Implementation]{Implementation}
This section describes the tools used to create the {\em NNGenerator} software. 
The language, libraries, and runtime used are described as well as the motivation for each. 
A few of the data structures used are also discussed.

% This document was generated automagically by lgrind.  DO NOT EDIT.

\section{Clojure}
The application is written in Clojure which is a LISP-like language for the JVM whose syntax is very similar to Scheme.   
Clojure combines many of the benefits of functional programming with the power of the JVM to allow for a very 
powerful tool for creating a distributed and multithreaded application such as this one. 
Clojure is fully interoperable with Java and so the parts of the application that interface with existing Java libraries are also written in Clojure.
Programmers that have used languages in the LISP family will be familiar with some concepts in Clojure
that are different enough to those coming from a C++/Java/C# background to be worth mentioning here.









 
The phrase "benefits of functional programming" is a somewhat arbitrary term that will be loosely defined by the few examples
discussed next. 

\subsection{Simplicity and Verbosity}
Functional programs take the programmer farther away from the hardware than non-functional ones. Consider the following simple example when asked to write a function that takes a list of integers as its input and outputs a list of integers of the same arity whose values are the incremented value of the corresponding value of the original list. So for example, given the input (-1 -2 -3 1 2 3) the output is (0 -1 -2 2 3 4). If asked to write this in C, Java, Pascal, or any other non-functional language, a typical implementation might look like the following Java code: 

% switching to Java
\LGinlinefalse\LGbegin\lgrinde
\index{increment}\Proc{increment}\L{\LB{\K{void}_\V{increment}(\K{int}[\,]_\V{x})_\{}}
\L{\LB{}\Tab{8}{\K{int}_\V{y}_=_\K{new}_\K{int}[\V{x}.\V{length}];}}
\L{\LB{}\Tab{8}{\K{for}(\K{int}_\V{i}_=_\N{0};_\V{i}_\<_\V{x}.\V{length};_\V{i}++)_\{_}}
\L{\LB{}\Tab{10}{\V{y}[\V{i}]=\V{x}[\V{i}]+\N{1};}}
\L{\LB{}\Tab{9}{\}}}
\L{\LB{}\Tab{9}{\K{return}_\V{y};}}
\L{\LB{\}}}
\endlgrinde\LGend

A typical Clojure implementation may look like the following: 

% switching to Scheme
\LGinlinefalse\LGbegin\lgrinde
\L{\LB{(\V{defn}_\V{inc\_func}_[\V{x}]_(\V{map}_\V{inc}_\V{x}))}}
\endlgrinde\LGend

Syntactically, obviously the Clojure version is much less verbose than the Java one.
Ignoring syntactical differences, one piece of code in the Java version that we see is uninteresting is the declaration,
checking, and incrementing of the variable 'i'. Here we must explicitly create and use a separate piece of state for indexing into the arrays. This temporary state is unnecessary
and can be replaced with Clojure's 'map' function. The 'map' function "returns a lazy sequence consisting of the result of applying f to the
set of first items of each coll, followed by applying f to the set
of second items in each coll, until any one of the colls is
exhausted". Map is a n-ary function, and when called with a single list, it applies an arbitrary function to every item in the list and returns the result. There is no 
temporary state to be able to index into the list. Although it can become natural to create and use temporary state in performing an operation on each item in a list of items,
it is usually not necessary to use the state in the computation, and it clutters the solution. 

Another uninteresting piece of code in the Java version is the line where y is defined. 
This declaration is really just an artifact of how John von Neumann based architectures work. In order to write this function without making changes to 
the original list, a new list of the same size must first be created, making the code unnecessarily verbose.


\subsection{Threads]
Clojure fully acknowledges the fact that mutable state is needed somewhere in most real world applications, however it does not use the problematic thread model to provide for reading and writing mutable state. Here we briefly examine a few of the problems with using threads. For a more thorough inspection refer to ~\cite{1076522}.

Problems arise if you do decide to edit a list inline either for terseness or to save memory and therefore compromise on the "without making changes to
the original list" constraint: 

% switching to Java
\LGinlinefalse\LGbegin\lgrinde
\index{incrementInPlace}\Proc{incrementInPlace}\L{\LB{\K{void}_\V{incrementInPlace}(\K{int}[\,]_\V{x})_\{}}
\L{\LB{}\Tab{8}{\K{for}(\K{int}_\V{i}_=_\N{0};_\V{i}_\<_\V{x}.\V{length};_\V{i}++)_\{_}}
\L{\LB{}\Tab{10}{\V{x}[\V{i}]++;}}
\L{\LB{}\Tab{9}{\}}}
\L{\LB{}\Tab{9}{\K{return}_\V{x};}}
\L{\LB{\}}}
\endlgrinde\LGend

For single threaded programs, there is no issue; in a multi-threaded environment, though, consider how the above code
causes issues even with just 2 threads calling it at the same time. If thread A attempts to access list at the same time as thread B is modified the list, then the result
that thread A computes is also invalid. It gets worse, the thread solution to this is to use a monitor or mutual exclusion block to ensure that no two threads can call the function at the
same time. In Java, the monitor can be implicit by using the method modifier 'synchronized' to the method definition. If the method is static,
the monitor is implicitly the singleton instance of an object's 'getClass' method:

% switching to Java
\LGinlinefalse\LGbegin\lgrinde
\index{foo}\Proc{foo}\L{\LB{\K{static}_\K{synchronized}_\K{void}_\V{foo}()_\{_\}}}
\endlgrinde\LGend

If the method is a non-static method of a class, the monitor is a 
particular instance of the class:

% switching to Java
\LGinlinefalse\LGbegin\lgrinde
\index{foo}\Proc{foo}\L{\LB{\K{synchronized}_\K{void}_\V{foo}()_\{_\}}}
\endlgrinde\LGend

You can also declare arbitrary objects and use them as monitors:
% switching to Java
\LGinlinefalse\LGbegin\lgrinde
\L{\LB{\V{Object}_\V{x}_=_\K{new}_\V{Object}();}}
\index{synchronized}\Proc{synchronized}\L{\LB{\K{synchronized}_(\V{x})_\{}}
\L{\LB{_\C{}//holding_monitor_of_x\CE{}}}
\L{\LB{\}}}
\endlgrinde\LGend

In C\#, the monitor is a library:
 
% switching to Java
\LGinlinefalse\LGbegin\lgrinde
\L{\LB{\V{System}.\V{Object}_\V{obj}_=_(\V{System}.\V{Object})\V{x};}}
\L{\LB{\V{System}.\V{Threading}.\V{Monitor}.\V{Enter}(\V{obj});}}
\L{\LB{\K{try}}}
\L{\LB{\{}}
\L{\LB{}\Tab{4}{\V{DoSomething}();}}
\L{\LB{\}}}
\L{\LB{\K{finally}}}
\L{\LB{\{}}
\L{\LB{}\Tab{4}{\V{System}.\V{Threading}.\V{Monitor}.\V{Exit}(\V{obj});}}
\L{\LB{\}_}}
\endlgrinde\LGend

With some syntactic sugar:

% switching to Java
\LGinlinefalse\LGbegin\lgrinde
\index{lock}\Proc{lock}\L{\LB{\V{lock}_(\V{x})}}
\L{\LB{\{}}
\L{\LB{}\Tab{4}{\V{DoSomething}();}}
\L{\LB{\}}}
\endlgrinde\LGend

Whether built into the language or provided as a library, locks create problems.
Consider what happens when thread A grabs a lock to object X and then waits for
a lock on object Y to be released, then gets interrupted, then thread B which is holding a lock
on object Y gets switched in and immediately begins waiting on the lock to object X, deadlock.
This makes it hard to create robust multithreaded API's, since the client code may grab the locks out of order not
knowing the lock ordering rules of the API. 

Another problem with locks is that when used in non-functional language they cause 
readers to block readers and writers, and writers to block readers and writers. Functional 
languages have no shared state, and so all functions are pure, meaning they have no observable
side effects. This is a very desirable property and it allows Clojure to provide a more optimistic version
of mutating shared state with the tradeoff being that some operations may have to run multiple times if run 
in a very contentious environment. 



\section{User Interface}
When the master module is started, a graphical user interface, or GUI, is displayed that allows for controlling various parameters of the application. 
One of these is the population size, which does not have to match the number of actual network slaves; it is the number of networks to train at each step. 
So if you have only eight slaves and you enter sixteen, then each slave will train two neural networks at each iteration of the search algorithm. 
Another parameter is the number of generations to train. 
As you increase this parameter, the time to complete the algorithm increases linearly. 
Another parameter sets the amount of iterations a slave should train a single neural network. 
This is a constant for all slaves so that each generated neural network has been given a fair chance at training its weights. 
The remaining parameters put upper bounds on the neural network structures themselves. 
A maximum number of hidden layers parameter sets an upper bound of the number of hidden layers for a neural network. 
A maximum number of nodes per layer parameter sets an upper bound on the number of nodes for a hidden layer. 
The number of input and output nodes depends on the map of input vectors to outputs used during training and cannot be changed by the user. 
Tweaking the number of iterations to train a network, the number of nodes per layer and number of layers can have a substantial effect on the time it takes for the algorithm to complete. 
To get the output of each hidden layer and the output layer is on the order of $NM$, or the multiplication between the weight matrix between the previous layer and the current layer and the nodes in that layer. 
As the number of nodes increases, the time to complete the operation for a layer increases.

The user interface is written using the Java Swing library~\cite{swing}. 
This library comes bundled with the Java Runtime Environment and allows creating cross-platform GUI's that look native to the OS or can be skinned in a customizable way. 
Swing uses a single threaded model and embraces the Model-View-Controller or MVC design pattern~\cite{mvc}. 
This pattern keeps the view code separate from the model code, and the view can save or update data the model only though the controller. 
In Swing, the controllers are action listeners which are fired based on events such as a mouse click or keyboard input. 
The action listeners all spawn a new thread to do their work, and later the UI may be updated asynchronously on the main GUI thread called the {\it Event Dispatch Thread} or EDT.
Unfortunately, the library cannot enforce this behavior, it is up to the programmer to keep this pattern in mind.
If work is done on the EDT that should be done in a background thread, the user interface will become slow and unresponsive. 

\section{Java Messaging Service}
Although the concurrency features of Clojure are nice, they are limited in that they only provide support for shared memory concurrency on a single JVM. 
The distributed architecture uses a messaging model that is facilitated via the Java Messaging Service or JMS API~\cite{jms, jms1}.
JMS is a Java API that provides applications with the ability to communicate via objects called messages.
A program that uses JMS to produce and/or consume messages is called a JMS client.
Message passing in JMS uses an asynchronous send in which a producer produces a message and does not block waiting for it to be consumed, though it does block waiting for the message broker to confirm receipt.
On the consumer side, receiving a message can either block until it has a message, or it can process messages asynchronously through an event listener.
In the latter case, an event will fire for every single message that the client receives.
JMS is also reliable, it ensures that every message successfully produced by a JMS client gets consumed by a subscriber if one is available.

There are two available approaches to messaging in JMS.
The first is a publish/subscribe model in which JMS clients publish messages to queues called {\it topics}.
A topic can have one or more subscribers and will send all messages in the topic to all current subscribers, meaning that a particular message may get processed more than once.
A client must be subscribed to a topic before it can receive messages from that topic.
This implies an ordering on the time a publisher publishes a message to a topic, and the time a subscriber consumes that message from the topic; the publish must happen first.
Figure \ref{pub_sub} is an illustrative example of the publish/subscribe model in JMS.

\begin{figure}[h!]
  \centering
  \includegraphics[scale=0.7]{images/pub_sub}
  \caption{JMS publish/subscribe model}
  \label{pub_sub}
\end{figure}

The second approach is a point-to-point model which is very similar to the publish/subscribe approach.
With this approach, messages are queued to named queues.
Each message will only get consumed once by a single JMS client.
There is no ordering on the when a client consumes a message from a queue and when a client publishes that message to the queue.
Figure \ref{point_to_point} shows how the point-to-point model works in JMS.

\begin{figure}[h!]
  \centering
  \includegraphics[scale=0.7]{images/point_to_point}
  \caption{JMS point-to-point model}
  \label{point_to_point}
\end{figure}

The {\em NNGenerator} software uses the point-to-point model.
We do not want either training messages or training result messages to ever be processed more than once.
The messaging model is simple, and as described in the architecture, there are two message queues. 
One queue is for the single master node to write to and for the slaves to read from, and the other is for the entire set of slaves to write to and for the master to read from. 
As soon as a master or slave node sends a message, it does not block waiting for a response. 

JMS message queues and topics are provided via JMS brokers.
All message queueing and dequeing is handled through one or more JMS brokers.
These brokers behave like proxies and have logic for making sure that messages get sent and received.
In a production environment, these brokers are usually replicated in case one of them fails.
The {\em NNGenerator} software expectes a single JMS broker to be available to the master and slaves nodes.
Implementors of the brokering services come in many different implementations. 
An implementation must implement the JMS specification for a broker in order to work with JMS clients.
During testing and while collecting results we are using a message broker provided by Apache ActiveMQ~\cite{activeMQ}.
The software could just as easily use another JMS broker to connect to. 
The ActiveMQ broker listens on a single TCP socket for messages from other nodes on the network. 
Nodes never have to talk directly to each other on the network; they only need to be able to connect to the broker.  
This allows the master and the slaves to operate on different networks and they never have to communicate directly.

The {\em NNGenerator} master and slave JMS clients both use asynchronous receives.
An event listener is wired up for both the slaves and master for when messages are received. 
The slave will sit idle until it has a training message to consume, at which point it will become CPU bound while it trains a neural network.
After it finishes training, it will sit idle again until another training message comes in.
The master also has a thread that sits idle and waits for result training messages in the same manner.
The master receives a message for each neural network that a slaves trains. 
It also receives heartbeat messages from slaves to determine when slaves disconnect, or might have run into some other problem that will prevent them from being able to train a network. 
Clojure multimethods are used to process incoming messages based on the type of the message. 
Slaves receive a different message type for each different problem. A problem is specified by its training data set. 
The type allows the slave to select the training data set to use for training the network. 
For example, the XOR input output map will be selected when a slave receives a {\em TRAIN-XOR} message. 
The training sets are distributed along with the slave jars to keep the master from having to send the data set over the network. 
The data can get very large in the case of binary formats such as images and sound. 
As stated in the architecture chapter, the algorithm can continue as long as one slave on the network is still able to receive messages. 
This is undesirable, however, as the algorithm is meant to fully exploit the inherent parallelism of genetic algorithms by having a one to one mapping between the number of slaves on the network and the size of the population. 

\section{Data Structures}
The neural network structures that are generated and bred are all backpropagation neural networks with at least one hidden layer. 
They are encoded as strings representing serialized maps in Clojure. 
These strings are relatively small compared to serialized Java objects and are also human readable since they are just Clojure data. 
The serializing and deserializing functions are also simple compared to what would be needed to serialize a Java object. Here is serialization: 

%#Scheme
%[
(defn serialize [x]
  (binding [*print-dup* true] (pr-str x)))
%]

and here is deserialization:

%#Scheme
%[
(defn deserialize [x]
  (let [r (new PushbackReader (new StringReader x))]
    (read r)))
%]

%TODO: example serialized structure

