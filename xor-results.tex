\chapter[XOR Problem]{XOR Problem}
The test problem that was used while building the application is a
simple {\it XOR} problem. 
The input/output map is provided to the program as the following
Clojure map: 

%#Scheme
%[
(def XOR-table {[-1 -1] [-1]
                [-1 1] [1]
                [1 -1] [1]
                [1 1] [-1]})
%]

The networks are trained with random samples from the map.

Table \ref{xor1} shows the results for running the {\it NNGenerator}
software with the following paramters: 

\begin{center}
\includegraphics[scale=0.7]{images/xparams_1}
\end{center}

\section{Results}
\begin{center}
    \begin{longtable}{ | l | l | l | l |}
      \caption{{\it XOR} Trainer Test Results 1} \label{xor1} \\
    \hline
    Generation & Lowest RMS Error & Average RMS Error \\ \hline
1 &	1.304085683943452E-27 &	0.027801054477356008 \\ \hline
2 &	7.596454196607839E-65 &	7.339520649571854E-8 \\ \hline
3 &	5.3432458075627907E-51 &	6.370854343870546E-11 \\ \hline
4 &	8.702628540823477E-45 &	8.464276299604785E-6 \\ \hline
5 &	1.0824533539549292E-27 &	5.308576218247767E-10 \\ \hline
6 &	1.5216521873927184E-53 &	1.1998367829509877E-9 \\ \hline
7 &	8.830083787845588E-50 &	1.5411988745508003E-9 \\ \hline
8 &	5.174072288071372E-47 &	3.11123830212436E-5 \\ \hline
9 &	4.930380657631324E-32 &	4.459504324391615E-12 \\ \hline
10 &	1.5307706768680235E-57 &	2.2127168768494285E-10  \\ \hline  
    \end{longtable}
\end{center}

It generated a neural network with one hidden layer as show in Figure \ref{xresults_1}.

\begin{figure}[h!]
  \centering
  \includegraphics[scale=0.5]{images/xresults_1}
  \caption{Resultant neural network for first {\it XOR} trainer result set}
  \label{xresults_1}
\end{figure}
