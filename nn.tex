\chapter[Neural Networks]{Neural Networks}
The purpose of an artificial neural network is to provide a mapping from a set of input data to output data. It builds this mapping in an iterative fashion from training data consisting of known input/output pairs that are presented to it. The training inputs are a subset of the total possible inputs to the network, and assuming there is a function of input to output data, a neural network is a black box that will approximate that function for us. Neural networks have been used as solutions in pattern recognition problems such as optical character recognition (OCR)\cite{ocr1}\cite{ocr2}, facial recognition\cite{face} as well as decision making problems\cite{decisionMaking1}\cite{decisionMaking2}.

The effectiveness of a neural network can be found by calculating it's error rate. A low error rate means the network is good at estimating the problem/solution function. Finding a network with a low error rate for a particular problem is challenging. One of the challenges comes from choosing which features of data to use to form inputs of a training set and how to represent this data. Another challenge is choosing how many hidden layers the network will have, how many nodes per hidden layer should be used, and what the activation function should be used in each hidden layer. The combination of these three things determines the network structure. 