\chapter[Genetic Algorithms]{Genetic Algorithms}
Genetic algorithms are search algorithms that attempt to find a generally good solution to a problem whose search space
is very large and therefore computationally inefficient to compute through enumeration or whose functions are discontinuous and cannot
be solved with methods requiring finding the derivative at points in the space. 

Genetic algorithms start with a random sample of encoded solutions to a problem. 
A fitness function for each sample is calculated which determines the fitness of a sample. 
A new generation of samples is created by applying simple operations to the samples. 
These are mutation, crossover, and reproduction. 
The ideas behind these operations are similar to natural selection mechanisms that occur during evolution, hence the name genetic algorithm. 
During mutation there is a low probability that a part of a sample in the population will be changed in some small way. 
If a binary encoded string 0100 were selected for example it may become 0101. 
Crossover combines two samples in the population at some randomly selected index called the cross site. 
For example, the string 00110 crossed with the string 10001 at index 2 will make the strings 10110 and 00001. 
Reproduction will copy the sample over to the new population based on some probability depending on the sample's fitness. 
In this way the fittest samples have the best chance of survival as in nature.

In the {\it NNGenerator} software, each chromosome has a probability of surviving proportional to itss fitness compared to the others in the same generation. 
So the least fit chromosomes have the least chance of reproduction and crossover when forming the next population of samples. 
This is implemented as roulette wheel selection where each item along the wheel is a separate chromosome and the probability of selecting a chromosome is directly proportional to its fitness.    

