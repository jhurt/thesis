\chapter[Blackjack Trainer]{Blackjack Trainer}
One of the problems used to test the program is a blackjack player. 
Blackjack is a card game whose basic premise to get a hand value that is closer to 21 than that of the dealer, without going over 21. 
Each player plays against the dealer and does not compare hands with other players. 
The players can only see the dealers top, or second dealt, card. 
The value of an Ace can count as either 1 or 11. 
The cards 2 through 10 are valued at their face value. 
The Jack, Queen, and King are all valued at 10. 
The players and dealer are initially dealt 2 cards each, and a player can choose to either hit one or more times in succession or to stay. 
Each time a player hits, the dealer deals him another card. 
When a player stays, it is either the next player's turn or the dealer's turn in the case all players have played. 
There are a few variations of rules in casinos that determine when a dealer should hit or stay. 
The most common, and the one used in this program, is that the dealer must hit until the combined value of his cards is greater than or equal to 17. 
There are also more variations on the options a player has in addition to hitting and staying, such as doubling down and splitting pairs. 
For simplicity of the trainer, these variations are not included in the blackjack trainer.

The input consists of a binary string of length nine.
The first five digits represent the value of the player's hand.
The last four digits represent the value of the card the dealer is
showing.
The value of the dealer's hand is unknown to the player in an actual blackjack game, so the dealer's first card is not used in the input. 
For example, if the players hand totals thirteen and the dealer's top
card is a three, the input string would be:  $011010011$.

The output pairs are found by playing simulated hands between a dealer and a single player. 
The output is 1 if hitting resulted in the player winning, -1 if
staying resulted in a win, and 0 otherwise.
The output is determined by using the dealer's rule of continuing to
hit until the value of the hand is greater than or equal to 17.
Note that this is not always the optimal strategy, therefore the
data used during training does not always train the network with the
optimal strategy.

The simulator used during training deals cards out of a shuffled deck
for each hand.
Because this shuffling is random, each slave will train its neural network structure with a different input/output map
than every other slave.
The chance that every possible combination of player's hand and dealer's top card
will be considered increases as the as the number
of training iterations increases. 

The trainer also has two simulators for testing a neural network.
One is a simulator that will play a specified number of games where
the player uses dealer's rules of hitting.
The second simulator also runs for a specified number of games using
the output of a neural network to determine whether or not to hit.
These two simulators are used to determine if a neural network
solution can do better than dealer's rules for hitting.

\section{Results}

wins:  7961 , ties:  1021 , losses:  41018

dealer wins:  20309 , ties:  9107 , losses:  20584


Table \ref{btr1} shows the results of running the {\it NNGenerator} software with the following parameters:

\begin{center}
\includegraphics[scale=0.7]{images/bparams_1}
\end{center}

\begin{center}
    \begin{longtable}{ | l | l | l |}
      \caption{Blackjack Trainer Test Results 1} \label{btr1} \\
   \hline
  Generation & Lowest RMS Error & Average RMS Error \\ \hline
1 &	0.0058351248061235105 &	0.008098956382062575 \\ \hline
2 &	0.005124453989806542 &	0.008788577240515025 \\ \hline
3 &	3.112780421164862E-4 &	0.006815992145352396 \\ \hline
4 &	0.001878277865251978 &	0.00615390146502372 \\ \hline
5 &	0.0033683109943283512 &	0.005027418011111019 \\ \hline
6 &	7.906376846118689E-26 &	0.0055728544464406365 \\ \hline
7 &	5.0558238025193206E-20 &	0.0013660235242617068 \\ \hline
8 &	2.669238353451669E-13 &	0.0019471455135151232 \\ \hline
9 &	7.581691939386478E-23 &	0.0013389815986515853 \\ \hline
10 &	5.125102775087759E-25 &	0.001963155659988856 \\ \hline
\end{longtable}
\end{center}

The resultant network has 18 hidden layers.
A screenshot of the resultant network is shown in Figure \ref{bresults_1}.

\begin{figure}[h!]
  \centering
  \includegraphics[scale=0.3]{images/bresults_1}
  \caption{Resultant neural network for first blackjack trainer result set}
  \label{bresults_1}
\end{figure}


