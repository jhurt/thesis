\chapter[Blackjack Trainer]{Blackjack Trainer}
One of the problems used to test the program is a blackjack player. Blackjack is a card game whose basic premise to get a hand value that is closer to 21 than that of the dealer, without going over 21. Each player plays against the dealer and does not compare hands with other players. The players can only see the dealers top, or second dealt, card. The value of an Ace can count as either 1 or 11. The cards 2 through 10 are valued at their face value. The Jack, Queen, and King are all valued at 10. The players and dealer are initially dealt 2 cards each, and a player can choose to either hit one or more times in succession or to stay. Each time a player hits, the dealer deals him another card. When a player stays, It is either the next players turn or the dealers turn in the case all players have played. There are a few variations of rules in casinoes that determine when a dealer should hit or stay. The most common, and the one used in this program, is that the dealer must hit until the combined value of his cards is greater than or equal to 17. There are also more variations on the options a player has in addition to hitting and staying, such as doubling down and splitting pairs. For simplicity of the trainer, these variations are not included in the blackjack trainer.

The input of a training pair for a blackjack player is the value of the cards the player is holding, plus the value of the single card the dealer is showing. The value of the dealer's hand is unknown to the player in an actual blackjack game, so the dealer's first card is not used in the input. The output pairs are found by playing simulated hands between a dealer and a single player. The output is 1 if hitting resulted in the player winning, -1 if staying resulted in a win, and 0 otherwise.






