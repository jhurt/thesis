\section{Clojure}
The application is written in Clojure~\cite{clj} which is a LISP-like language for the Java Virtual Machine~\cite{jvm}, or JVM. 
Clojure is fully interoperable with Java and so the parts of the application that interface with existing Java libraries are also written in Clojure.
Programmers that have used languages in the LISP family may be familiar with some concepts that are present in Clojure, such as
functions as arguments, dynamic typing and lazy evaluation. 
Other features present in Clojure used in {\em NNGenerator} are interesting enough to be worth mentioning here also, including multimethods and software transaction memory.

\subsection{Functions as Data Types}
In functional languages, functions are first class data types; they
can be passed to other functions as arguments, thereby allowing specific behavior not related to the calling function to be coded
separately from the calling function. 
Object oriented languages accomplish similar behavior with polymorphism which is
facilitated through inheritance. 
Having functions as first class data types in a language is more
flexible than inheritance because the runtime only needs to check a
function's signature and does not need to examine the type hierarchy
that a function belongs to.   

\subsection{Closures}
The name for Clojure is based on the term {\it closure}, which
describes a function that has access to all variables bound in the
closure's current scope. 
In the case of a closure, in addition to a function definition, pieces
of state that are not in global scope can be optionally bound. 
In this a function can be created that uses state that is hidden
from the function that invokes it.
A simple example of this concept is the following function {\it incrementAndGet}.
It keeps track of a variable to increment and return without using
global state: 

%#Scheme
%[
(let [i (ref 0)]
  (defn incrementAndGet []
     (dosync (ref-set i (inc @i))) @i))
%]

Here is the example output from calling this multiple times in a
Clojure REPL:

%#Scheme
%[
user=> (incrementAndGet)
1
user=> (incrementAndGet)
2
user=> (incrementAndGet)
3
%]
 
\subsection{Dynamic Typing}
Java is a statically typed language; its function arguments are
checked at compile time and the programmer must explicitly mark all
object references with a
type declaration. 
This can help to catch errors in calling functions at compile time
rather than at run time, but the code becomes littered with types, and
changing the structure of a type or adding functionality to an object
is difficult. 
    
Clojure is a dynamically typed language; it determines the actual type
of method parameters at runtime, and it accomplishes this via the Java
Reflection Library~\cite{reflection}. 
In the rare case that reflection is causing performance issues, such
as a tight CPU bound loop, type hints can be provided for variables. 
In Java 7, a new bytecode instruction called
{\tt invokedynamic}~\cite{invokedynamic} has been added.
This instruction allows dynamic
languages on the JVM to have native support for dynamic typing without
using the Reflection library. 
This instruction is similar to the {\tt invokevirtual} instruction which is
used whenever a virtual method is called in Java. 
By default all methods in Java are virtual unless marked as private or static. 
This is because static and private methods cannot be overridden in a
subclass, so there is no need to lookup their location at runtime
since it can be hardcoded at compile time. 
The addition of the invokevirtual instruction will allow future Clojure implementations to see performance improvements as opposed to using reflection, which is much slower in comparison.  

\subsection{Lazy Evaluation}
Clojure supports lazy evaluation, meaning that program statements are not
evaluated line by line as they are with imperative languages. 
The result of any particular computation is not actually computed until the
exact moment it is needed.
The result may be needed in order to evaluate the result of another computation. 
It may also be needed in a method that has side effects, such as
writing data to a socket or writing some data to a video buffer, or in
the case of Clojure/Java interop where the Java functions are not
typically pure and need to be executed to force some side effect.
This lack of pureness in Java functions is typical in object oriented
program and means that most methods are not idempotent.
In other words, calling the same set of functions repeatedly with the
same arguments does not necessarily produce the same result.
This makes reasoning about the code particularly hard.
The result may never actually be needed in the course of the program
and in this case it will never be calculated.
Special syntax is provided for when it is necessary for the programmer
to force
evaluation. The following in a snippet from {\em NNGenerator} that
uses the {\it doall} keyword to force the side effect of the {\it rmdir} Java method: 

%#Scheme
%[
(defn rmdir [dir]
  (if (.isDirectory dir)
      (doall (map rmdir (.listFiles dir))))
  (.delete dir))
%]

Lazy evaluation allows Clojure to represent things such as infinite sequences,
which are otherwise very hard to represent in imperative languages. 
The following defines the infinite set of positive even integers: 

%#Scheme
%[
(def evens (iterate (fn [x] (+ 2 x)) 0))
%]

Obviously realizing this full set at once is not possible because it
is an infinite set and the machine has a finite memory.
Parts of this set can be realized in smaller pieces. Here is some
output of using the {\it take} method to fetch pieces of the set: 
%#Scheme
%[
user=> (take 4 evens)
(0 2 4 6)
user=> (take 10 evens)
(0 2 4 6 8 10 12 14 16 18)
user=> (take 15 evens)
(0 2 4 6 8 10 12 14 16 18 20 22 24 26 28)
%]

A particular downside of laziness is that it makes debugging harder. 
While stepping through code line by line, some expressions may not
ever be calculated even though the code for the expression has already
been seen by the runtime.

Another interesting case of lazy evaluation is that it is possible to
blow the call stack if the level of unevaluated expressions gets too
deep. While writing the {\em NNGenerator} software this happened in
the Slave module for large training sets. The matrix multiplication
calls were not being evaluated since the result was not needed until
the training of the entire neural network was complete.
To circumvent this issue, the {\it doall} is used in the {\it
  matrixAdd} method:
 
%#Scheme
%[
(defn matrixAdd [matrixA matrixB]
  (if (and (not (empty? matrixA)) (not (empty? matrixB)))
    (conj
      (matrixAdd (rest matrixA) (rest matrixB))
      (doall (map + (first matrixA) (first matrixB))))))
%]

\subsection{Multimethods}
Many object oriented languages such as Java and C\# support dynamic
method dispatch through a concept called polymorphism. 
The runtime selects an appropriate method to call for an object based
on the runtime type of the object. 
One reason for this is to be able to pass functionality into methods via interface type declarations because the languages do not support functions as first class objects. 
Clojure goes beyond polymorphism and offers a more general concept of dynamic method dispatch in which the programmer defines
the dispatch function as opposed to always using the runtime type of a particular object. 

The following uses polymorphism in Java:

%#Java
%[
interface Foo { 
	String foo(); 
}

class FooBar implements Foo {
        public String foo { return "foobar"; }
}

class FooCat implements Foo {
        public String foo { return "foocat"; }
}
%]

To implement polymorphism in Clojure using the same Java types:

%#Scheme
%[
(defmulti foo (fn [obj] (.getName (.getClass obj))))

(defmethod foo "FooBar" [obj] "foobar")

(defmethod foo "FooCat" [obj] "foocat")
%]

The {\it defmulti} macro is used to define the name of the function, in this
case {\it foo}. It also defines the dispatch function for the argument
or arguments that gets passed to the function. In this case, a single
argument will be passed to the function, and the classname of the
argument will be returned by the dispatch function.
The {\it defmethod} macros create and install two new methods of
multimethod {\it foo} associated with the dispatch-values {\it FooBar}
and {\it FooCat}. 

The following is a switch statement in Java:
%#Java
%[
MyCalendar cal = new MyCalendar();
switch (cal.getMonth()) {
            case 1:  System.out.println("January"); break;
            case 2:  System.out.println("February"); break;
            case 3:  System.out.println("March"); break;
            case 4:  System.out.println("April"); break;
            case 5:  System.out.println("May"); break;
            case 6:  System.out.println("June"); break;
            case 7:  System.out.println("July"); break;
            case 8:  System.out.println("August"); break;
            case 9:  System.out.println("September"); break;
            case 10: System.out.println("October"); break;
            case 11: System.out.println("November"); break;
            case 12: System.out.println("December"); break;
            default: System.out.println("Invalid month.");break;
        }
%]

This can be broken up into many methods using a multimethod in Clojure:

%#Scheme
%[
(defmulti getMonthString (fn [cal] (.getMonth cal)))

(defmethod getMonthString 1 [cal] "January")
(defmethod getMonthString 2 [cal] "February")
(defmethod getMonthString 3 [cal] "March")
(defmethod getMonthString 4 [cal] "April")
(defmethod getMonthString 5 [cal] "May")
(defmethod getMonthString 6 [cal] "June")
(defmethod getMonthString 7 [cal] "July")
(defmethod getMonthString 8 [cal] "August")
(defmethod getMonthString 9 [cal] "September")
(defmethod getMonthString 10 [cal] "October")
(defmethod getMonthString 11 [cal] "November")
(defmethod getMonthString 12 [cal] "December")
(defmethod getMonthString :default [cal] "Invalid month.")
%]

If simply writing a String is all that a switch statement does, the advantage of breaking it up into many
methods is not noticeable. For complex logic that would typically be handled with a long else if or a switch statement, 
multimethods help keep code clean and more readable.

\subsection{Simplicity and Verbosity}
Functional programs take the programmer farther away from the hardware
than non-functional ones. 
Consider the following simple example when asked to write a function
that takes a list of integers as its input and outputs a list of
integers of the same arity whose values are the incremented value of
the corresponding value of the original list. 
For example, given the input (-1 -2 -3 1 2 3), the output is (0 -1
-2 2 3 4). 
If asked to write this in C, Java, Pascal, or any other non-functional language, a typical implementation might look like the following Java code: 

%#Java
%[
int[] increment(int[] x) {
	int[] y = new int[x.length];
	for(int i = 0; i < x.length; i++) { 
	  y[i]=x[i]+1;
         }
         return y;
}
%]

A typical Clojure implementation may look like the following: 

%#Scheme
%[
(defn inc_func [x] (map inc x))
%]

Syntactically, the Clojure version is much less verbose than the Java one.
Ignoring syntactical differences, one piece of code in the Java version that we see is uninteresting is the declaration,
checking, and incrementing of the variable {\tt i}. 
Here we must explicitly create and use a separate piece of state for indexing into the arrays. 
This temporary state is unnecessary and can be replaced with Clojure's
{\it map} function. The {\it map} function returns a lazy sequence
consisting of the result of applying {\it f} to the
set of first items of each collection, followed by applying {\it f} to the set
of second items in each collection, until any one of the collections is
exhausted. {\it map} is a n-ary function, and when called with a
single list, it applies a function {\it f} to every item in the list
and returns the result. 
There is no temporary state to be able to index into the list. 
Although it can become natural to create and use temporary state in performing an operation on each item in a list of items,
it is usually not necessary to use the state in the computation, and it clutters the solution. 

Another uninteresting piece of code in the Java version is the line
where {\tt y} is defined. 
This declaration is really just an artifact of how John von Neumann based architectures work. In order to write this function without making changes to 
the original list, a new list of the same size must first be created, making the code unnecessarily verbose. 

\subsection{State}
Many functional languages are called pure functional languages. 
They are "pure" in the sense that every function takes some state and
returns some state and does not modify any global state. 
In fact there is no global state. 
This makes reasoning about code and verifying certain properties much
simpler than a procedural or object oriented language where state is
everywhere and functions can do whatever they want with that
state~\cite{process}. 
In fact, these languages do not even enforce correctness when working
with state from multiple threads; programmers are forced to come up
with correct solutions to these problems themselves.
 All of the data types native to Clojure support the concept of
 purity. 
When you do operations on lists for example, the results of
 those operations are actually new lists and the original one passed
 in remains unchanged. 
This has the same semantics as call by value with performance
comparable to call by reference without the danger of destroying the
reference. 
In some LISP implementations this causes performance problems
since lists are getting copied around a lot. 
Clojure uses a much more efficient way of doing this, which underneath the hood shares the data
structures~\cite{cljDataStructures} and so can give the same
performance guarantees as using mutable data structures in Java
without the added complexity of mutable state. 
Another problem with state and object oriented languages is misuse of
mutability. 
By default, member variables are mutable unless declared as final by
the programmer. 
Even the final modifier only enforces that the referenced object
cannot be changed after being set in the constructor; it does not
enforce any immutability with respect to the object's data members. 
In Clojure, the opposite is true, every data definition is immutable
by default unless it is explicitly made not mutable by the
programmer. 
When defining a Java data type, it cannot enforce anything with regard
to mutating that object's state, and this is one thing to watch out
for when using the Clojure/Java interop feature. 
Although lack of global and mutable state can be nice for theorem proving and code reasoning, in the real world programs that do something useful usually have some mutable state somewhere and are not just collections of pure functions and immutable data.  

\subsubsection{Threads}
Clojure fully acknowledges the fact that mutable state is needed
somewhere in most real world applications, however, it does not use
the problematic thread model to provide for reading and writing
mutable state. 
Here we briefly examine a few of the problems with using threads. 
For a more thorough inspection refer to ~\cite{1076522}.

Problems can arise if one decides to edit a list inline either for terseness or to save memory and therefore compromise on the 
"without making changes to the original list" constraint: 

%#Java
%[
int[] incrementInPlace(int[] x) {
	for(int i = 0; i < x.length; i++) { 
	  x[i]++;
         }
         return x;
}
%]

For single threaded programs, there is no issue; in a multi-threaded environment, though, consider how the above code
causes issues even with just two threads calling it at the same time. 
If thread $A$ attempts to access list at the same time as thread $B$ is modified the list, then the result
that thread $A$ computes is also invalid. 
It gets worse; the thread solution to this is to use a monitor or mutual exclusion block to
ensure that no two threads can call the function at the same time. 
In Java, the monitor can be implicit by using the method modifier {\tt synchronized} to the method definition. 
If the method is static, the monitor is implicitly the singleton instance of an object's {\it getClass} method:

%#Java
%[
static synchronized void foo() { }
%]

If the method is a non-static method of a class, the monitor is a 
particular instance of the class and is shared among all non-static synchronized methods of a particular object:

%#Java
%[
synchronized void foo() { }
%]

You can also declare arbitrary objects and use them as monitors:
%#Java
%[
Object x = new Object();
synchronized (x) {
 //holding monitor of x
}
%]

In C\#, the monitor is a library:
 
%#Java
%[
System.Object obj = (System.Object)x;
System.Threading.Monitor.Enter(obj);
try
{
    DoSomething();
}
finally
{
    System.Threading.Monitor.Exit(obj);
} 
%]

With some syntactic sugar:

%#Java
%[
lock (x)
{
    DoSomething();
}
%]

Whether built into the language or provided as a library, locks create problems~\cite{problemsWithThreads}.
Consider what happens when thread $A$ grabs a lock to object $X$ and then waits for
a lock on object $Y$ to be released, then gets interrupted, then thread $B$, which is holding a lock
on object $Y$, gets switched in and immediately begins waiting on the lock to object $X$, deadlock.
This makes it hard to create robust multithreaded API's, since the client code may grab the locks out of order not
knowing the lock ordering rules of the API. 

Another problem with locks is that, when used in non-functional language, they cause 
readers to block readers and writers, and writers to block readers and
writers. 
All of this blocking leads to performance problems when many threads are reading or writing some shared state.

\subsubsection{Software Transactional Memory}
Software transactional memory, or STM, is a pattern that is used for
providing transactional memory in a distributed shared
memory~\cite{stm}. 
STM in Clojure is completely non-blocking. 
Readers do not block readers, and writers do not block readers. 
Readers read the value of a reference at the time requested; in other
words, the latest committed value. 
Writers do not block writers, and readers do not block writers. 
Many writers can run in parallel, and, before they commit their
transactions, they check the value of the shared state they
are going to write to. 
If the value has changed since the start of the transaction, the
writer does its work again with the new value of the shared state. 
In this way, writers do not block each other, and high contention to
write a particular value results in writers repeating their work more
than once. 
In Java, this would prove difficult as most functions are not pure and cannot be run repeatedly in transactions blindly. 

Another benefit of STM in Clojure is that nested transactions are
handled without other functions knowing the order and nature of the
transactions. 
As previously described, using the thread model in Java, locks shared
among classes must be acquired in a certain order to avoid deadlocking
multiple threads. 
The correct use and/or the knowledge of this ordering is often missing
from programs. 
Providing the ordering couples classes together in a way that is not necessary with Clojure's STM.  

Clojure combines many of the benefits of functional programming with the power of the JVM to allow for a very powerful tool for creating a distributed and multithreaded application such as this one.  





